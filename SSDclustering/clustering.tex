\newpage
\section{Clustering}
There are two distinct set of approaches to our clustering:
\begin{enumerate}
\item cluster the input images
  \begin{enumerate}
   \item simple region covariance \ref{regcov}
   \item some form of non-Local Means (NLM)
\end{enumerate}
\item cluster the feature vectors of a pre-trained model
\end{enumerate}

There the algorithm to perform the clustering also deserve some investigation.
K-means is certain a starting point but it is very sensitive to initialization.
Next, the question of producing a quasi balanced cluster should also be considered.

If we are using a NLM approach, naive approach will make $k$ passes for $k$ clusters
while testing for nearest center. At least we need a fast lookup structure so 
that the nearest center can be retrieved in $O(logn)$ or $O(1)$ time.
Or use the very nice SigmaSet \ref{sigmaset} or \cite{Kwatra2010}

\subsection{Region Covariance}
Region covariance can be computed very efficiently using 'integral images/sumarea table" \cite{Porikli2006}.

\href{https://spie.org/news/0368-using-covariance-improves-computer-detection-and-tracking-of-humans?SSO=1}{Using covariance improves computer detection and tracking of humans}

\subsubsection{\cite{Tuzel2006}}\label{regcov}
"Region covariance: A fast descriptor for detection and classification"

\subsubsection{Sigma set}\label{sigmaset}
\cite{Chang2009} 

\subsection{Kwatra2010}
"Fast Covariance Computation and
Dimensionality Reduction for Sub-Window Features in Images"

\subsubsection{Faulkner2015}
\cite{Faulkner2015} "A Study of the Region Covariance Descriptor: Impact of Feature Selection and Image Transformations"

\subsection{Non-Local Means}
\subsubsection{Qian2013}
\cite{Qian2013} 
nonlocal similarity and spectral-spatial structure of hyperspectral imagery into sparse representation. Non-locality means the self-similarity of image, by which a whole image can be partitioned into some groups containing similar patches. The similar patches in each group are sparsely represented with a shared subset of atoms in a dictionary making true signal and noise more easily separated.

\subsubsection{Fu2017}
\cite{Fu2017}